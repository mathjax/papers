% !TEX TS-program = pdflatex
% !TEX encoding = UTF-8 Unicode

\documentclass{article}

\usepackage{graphicx}
\usepackage{verbatim}
\usepackage{hyperref}

\title{MathJax: the Present and the Future}

\author{Davide Cervone and Volker Sorge}
\date{July 20, 2020}

\begin{document}
\maketitle

\noindent At the Joint Mathematics Meetings in January of 2010, we announced
version 1.0 of the MathJax Javascript library for displaying typeset
mathematics in web pages.  Almost immediately, MathJax became the
de-facto standard for including mathematics on the web.  MathJax is used in
a wide range of on-line journals, including the AMS's own MathSciNet~\cite{MathSciNet}
website; it enables on-line blogs, wikis, and
question-and-answer sites like StackExchange~\cite{StackExchange} and
Wikipedia~\cite{Wikipedia} to include mathematical expressions; it
provides the mathematics for on-line homework systems like WeBWorK
\cite{WeBWorK} and learning management systems like Moodle
\cite{Moodle}; and it has been incorporated into e-book readers,
screen readers, and similar products~\cite{MJ-use}.

In the ten years since its introduction, MathJax has expanded
to include {\LaTeX}, MathML, and AsciiMath input formats
\cite{MJ-input}, and several output formats~\cite{MJ-output}, including HTML-with-CSS,
SVG (scalable vector graphics), and MathML.  The
introduction of sophisticated semantic enrichment and
speech-generation functionality in June 2016~\cite{MJ-accessibility}
has made MathJax a crucial component in generating web pages and
e-books that are accessible to readers with visual impairments who use
assistive technology like screen readers or Braille output devices.
MathJax can make on-line course materials or published research
accessible, which is of growing importance in this age of distance
learning.

Much has changed in web technology since MathJax was first introduced.
New web libraries and improvements to the Javascript language itself
have changed the way web-page designers want to use MathJax, and some
of the approaches built into MathJax in its early days made it hard to
include MathJax in modern web-page workflows.  For the past three years,
MathJax has been undergoing a complete rewrite from the ground up
with the goal of modernizing MathJax's internal infrastructure,
bringing it more flexibility for use with contemporary web
technologies, making it easier to use for pre-processing and
server-side support, and making the production of typeset mathematics
faster.  The release of MathJax version 3.0 in August 2019 brought
these hopes to fruition~\cite{MJ3}.

In order to make MathJax easier to maintain, version 3 is written in the
Typescript language~\cite{Typescript}, a version of Javascript that allows
types to be added to variables and functions that can be checked by a compiler for
correctness.  This enables errors to be found earlier, which leads to more
reliable code that is easier to understand, consequently making it easier for
others to contribute to MathJax.  It also lets us use new features of
Javascript that are part of the latest ES6 standard~\cite{ES6}, while still
supporting older browsers that do not implement them.
So, for example we now use ES6's modern class structure~\cite{ES6-class}
and take advantage of asynchronous features like promises~\cite{ES6-promise}.
As these were not available ten years ago, v2 implemented its
own object system and provided custom signals, queues, and callbacks
for asynchronous operation.

Version 3's modern ES6 features, which can be exploited by today's Javascript
interpreters for run-time optimization, together with its new internal
structure, remove performance issues that were inherent in the
design of v2, improving the rendering speed of MathJax.  Because the
two versions operate so differently, it is difficult to make precise
comparisons, but in tests that render a complete page with several
hundred expressions, we see a reduction in rendering time of between
60 and 80 percent, depending on the browser and operating system.

MathJax v2 used its own loading mechanism for accessing its
components, which did not work well with modern Javascript packaging
systems like webpack~\cite{webpack} or rollup~\cite{rollup}.  Version~3
resolves that problem, so it can interoperate better with modern web
workflows: one can make custom single-file builds of MathJax, for example, or
include it as one component of a larger asset file.

New in v3 is the ability to run MathJax synchronously. In
particular, v3 provides functions that can translate an input string
(say a TeX expression) into an output DOM tree (say an SVG
image)~\cite{MJ-convert}, and these can be applied to individual
expressions or entire documents.  This is particuarly important for
the preparation of rendered pages for off-line consumption, but was
not easy in version 2 since its operation was inherently asynchronous.

MathJax was designed originally for use in a web browser, but this
left the desire to pre-process mathematics on a
server unaddressed. MathJax~v3 was redesigned to make this possible, as it can be
used within node applications in essentially the same way as in a
browser~\cite{MJ-web, MJ-node}. That is, you can load MathJax
components, configure them through the MathJax global variable, and
call the same functions for typesetting and conversion as you do
within a browser. This makes parallel development for both the browser
and server much easier.  Moreover, node applications can access
MathJax modules directly (without the packaging needed for MathJax's
use in a browser). This gives the most direct access to MathJax’s
features, and the most flexibility in controlling MathJax’s actions.

Making mathematics accessible to readers with disabilities has been an
important feature that distinguishes MathJax from other math-rendering
solutions. While originally aimed to work with third-party assistive-technology
software of the time, since version 2.7 MathJax also has offered
its own accessibility solution that is meant to operate independently of
the browser, operating system, and assistive technology solutions (e.g.,
screen readers) that a user employs. The extension offers support for readers with
special needs, in particular those with visual impairments (e.g.,
blindness or low vision) or cognitive impairments (e.g., dyslexia or
dyscalculia).

The accessibilty extension uses the speech rule engine
library~\cite{SRE} for translating mathematical expressions into speech strings.
Its core features comprise the automatic voicing of formulas together
with their interactive navigation and synchronised highlighting, and an
abstraction feature that allows for visual simplification and summary
speech description of formulas.

Not only have these features been
ported to MathJax v3, but they have been greatly extended, making use of its new
modular setup to allow for a flexible pick-and-mix style
personalization of accessibility tools.
In particular, MathJax now offers a number of different rule sets for
voicing mathematics that can be pre-selected or switched on the
fly. That is, during interactive exploration of an expression, a reader can choose
another rule set, resulting in the expression being spoken in a different way,
thus providing a new view of it.  Low-vision
support has been improved by enabling magnification not only for an
entire formula, but also for sub-expressions during interaction. A
further emphasis is to provide better accessibility to advanced
mathematical material by exploiting information gained from the original
{\LaTeX} code to generate more appropriate speech for different areas of
Mathematics, as well as for subjects like Physics, Chemistry and Logic.

Although MathJax originally was intended as a stop-gap measure until
browsers implemented native math rendering (through MathML), after
more than ten years, browser support for MathML is not universal, and
MathJax continues to bring quality math typesetting to all modern
browsers.  Our work has been supported by grants from the Sloan
Foundation and the Simon's foundation, as well as generous
contributions from our sponsors, including the AMS, SIAM, Elsevier,
IEEE, and a variety of professional societies, publishers, and web
sites~\cite{MJ-sponsors}, without whose ongoing financial support,
MathJax would not have been possible.  The version 3 rewrite puts
MathJax in a strong position to continue to make beautiful and
accessible mathematics available on the web for years to come.

\small
\begin{thebibliography}{99}

\bibitem{ES6}
  ECMAScript 2020, \url{http://ecma-international.org/ecma-262/}.

\bibitem{ES6-class}
  ES6 classes, \url{https://developer.mozilla.org/Web/JavaScript/Reference/Classes}.
  
\bibitem{ES6-promise}
  ES6 promises, \url{https://developer.mozilla.org/Web/JavaScript/Reference/Global\_Objects/Promise}.

\bibitem{MJ}
  MathJax home page, \url{https://www.mathjax.org}.

\bibitem{MJ3}
  MathJax version 3, \url{https://www.mathjax.org/MathJax-v3-0-5-available/}.

\bibitem{MJ-accessibility}
  MathJax accessibility features, \url{https://docs.mathjax.org/en/latest/basic/accessibility.html}.

\bibitem{MJ-use}
  MathJax in use, \url{https://docs.mathjax.org/en/v2.7-latest/misc/mathjax-in-use.html}.

\bibitem{MJ-input}
  MathJax input formats, \url{https://docs.mathjax.org/en/latest}.
  
\bibitem{MJ-node}
  MathJax node demos, \url{https://github.com/mathjax/MathJax-demos-node\#mathjax-demos-node}.

\bibitem{MJ-output}
  MathJax output formats, \url{https://docs.mathjax.org/en/latest/output/index.html}.

\bibitem{MJ-sponsors}
  MathJax Sponsors, \url{https://www.mathjax.org/\#sponsors}.

\bibitem{MJ-convert}
  MathJax typesetting commands, \url{https://docs.mathjax.org/en/latest/web/typeset.html}.

\bibitem{MJ-web}
  MathJax web demos, \url{https://github.com/mathjax/MathJax-demos-web\#mathjax-demos-web}.

\bibitem{MathSciNet}
  MathSciNet, \url{https://www.ams.org/mathscinet/}.

\bibitem{Moodle}
  Moodle, \url{https://moodle.org}.

\bibitem{rollup}
  Rollup, \url{https://rollupjs.org/guide/en/}.

\bibitem{StackExchange}
  StackExchange, \url{https://stackexchange.com/}.

\bibitem{SRE}
  Speech Rule Engine, \url{https://speechruleengine.org}.

\bibitem{Typescript}
  Typescript, \url{https://www.typescriptlang.org}.

\bibitem{webpack}
  Webpack, \url{https://webpack.js.org}.

\bibitem{WeBWorK}
  WeBWorK, \url{https://webwork.maa.org}.

\bibitem{Wikipedia}
  Wikipedia, \url{https://www.wikipedia.org}.

\end{thebibliography}


\end{document}


